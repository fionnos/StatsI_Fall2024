\documentclass[12pt,letterpaper]{article}
\usepackage{graphicx,textcomp}
\usepackage{natbib}
\usepackage{setspace}
\usepackage{fullpage}
\usepackage{color}
\usepackage[reqno]{amsmath}
\usepackage{amsthm}
\usepackage{fancyvrb}
\usepackage{amssymb,enumerate}
\usepackage[all]{xy}
\usepackage{endnotes}
\usepackage{lscape}
\newtheorem{com}{Comment}
\usepackage{float}
\usepackage{hyperref}
\newtheorem{lem} {Lemma}
\newtheorem{prop}{Proposition}
\newtheorem{thm}{Theorem}
\newtheorem{defn}{Definition}
\newtheorem{cor}{Corollary}
\newtheorem{obs}{Observation}
\usepackage[compact]{titlesec}
\usepackage{dcolumn}
\usepackage{tikz}
\usetikzlibrary{arrows}
\usepackage{multirow}
\usepackage{xcolor}
\newcolumntype{.}{D{.}{.}{-1}}
\newcolumntype{d}[1]{D{.}{.}{#1}}
\definecolor{light-gray}{gray}{0.65}
\usepackage{url}
\usepackage{listings}
\usepackage{color}

\definecolor{codegreen}{rgb}{0,0.6,0}
\definecolor{codegray}{rgb}{0.5,0.5,0.5}
\definecolor{codepurple}{rgb}{0.58,0,0.82}
\definecolor{backcolour}{rgb}{0.95,0.95,0.92}

\lstdefinestyle{mystyle}{
	backgroundcolor=\color{backcolour},   
	commentstyle=\color{codegreen},
	keywordstyle=\color{magenta},
	numberstyle=\tiny\color{codegray},
	stringstyle=\color{codepurple},
	basicstyle=\footnotesize,
	breakatwhitespace=false,         
	breaklines=true,                 
	captionpos=b,                    
	keepspaces=true,                 
	numbers=left,                    
	numbersep=5pt,                  
	showspaces=false,                
	showstringspaces=false,
	showtabs=false,                  
	tabsize=2
}
\lstset{style=mystyle}
\newcommand{\Sref}[1]{Section~\ref{#1}}
\newtheorem{hyp}{Hypothesis}

\title{Problem Set 3}
\date{Due: November 11, 2024}
\author{Applied Stats/Quant Methods 1}


\begin{document}
	\maketitle
	\section*{Instructions}
	\begin{itemize}
		\item Please show your work! You may lose points by simply writing in the answer. If the problem requires you to execute commands in \texttt{R}, please include the code you used to get your answers. Please also include the \texttt{.R} file that contains your code. If you are not sure if work needs to be shown for a particular problem, please ask.
	\item Your homework should be submitted electronically on GitHub.
	\item This problem set is due before 23:59 on Sunday November 11, 2024. No late assignments will be accepted.

	\end{itemize}

		\vspace{.25cm}
	
\noindent In this problem set, you will run several regressions and create an add variable plot (see the lecture slides) in \texttt{R} using the \texttt{incumbents\_subset.csv} dataset. Include all of your code.

	\vspace{.5cm}
\section*{Question 1}
\vspace{.25cm}
\noindent We are interested in knowing how the difference in campaign spending between incumbent and challenger affects the incumbent's vote share. 
	\begin{enumerate}
		\item Run a regression where the outcome variable is \texttt{voteshare} and the explanatory variable is \texttt{difflog}.	
		
		\lstinputlisting[language=R, firstline=44, lastline=45]{PS3_FOS.R}  
		\begin{verbatim}
		Coefficients:
		(Intercept)      difflog  
				0.57903      0.04167  
		\end{verbatim}
		\vspace{2cm}
		\item Make a scatterplot of the two variables and add the regression line. 	
		\lstinputlisting[language=R, firstline=47, lastline=49]{PS3_FOS.R} 
		\begin{figure}[h!]
			\caption{\footnotesize{Model 1 }}
			\vspace{.5cm}
			\centering
			\label{fig:Model 1"}			
			\includegraphics[width=1.1\textwidth]{model1.png}
		\end{figure}		
		\vspace{2cm}
		\item Save the residuals of the model in a separate object.	
		\lstinputlisting[language=R, firstline=59, lastline=60]{PS3_FOS.R} 
		\vspace{2cm}
		\item Write the prediction equation.
		\lstinputlisting[language=R, firstline=63, lastline=64]{PS3_FOS.R} 
	\end{enumerate}
	\vspace{.5cm}
\newpage

\section*{Question 2}
\noindent We are interested in knowing how the difference between incumbent and challenger's spending and the vote share of the presidential candidate of the incumbent's party are related.	\vspace{.25cm}
	\begin{enumerate}
	\item Run a regression where the outcome variable is \texttt{presvote} and the explanatory variable is \texttt{difflog}.	
	
	\lstinputlisting[language=R, firstline=69, lastline=69]{PS3_FOS.R}  
	\begin{verbatim}
    Coefficients:
	(Intercept)      difflog  
			0.50758      0.02384  
	\end{verbatim}
	\vspace{2cm}
	\item Make a scatterplot of the two variables and add the regression line. 	
	\lstinputlisting[language=R, firstline=71, lastline=72]{PS3_FOS.R} 
	\begin{figure}[h!]
		\caption{\footnotesize{Model 2}}
		\vspace{.5cm}
		\centering
		\label{fig:Model 2}			
		\includegraphics[width=1.1\textwidth]{model2.png}
	\end{figure}		
	\vspace{2cm}
	\item Save the residuals of the model in a separate object.	
	\lstinputlisting[language=R, firstline=82, lastline=82]{PS3_FOS.R} 
	\vspace{2cm}
	\item Write the prediction equation.
	\lstinputlisting[language=R, firstline=84, lastline=84]{PS3_FOS.R} 
\end{enumerate}
	
	\newpage	
\section*{Question 3}

\noindent We are interested in knowing how the vote share of the presidential candidate of the incumbent's party is associated with the incumbent's electoral success.
	\vspace{.25cm}
	\begin{enumerate}
		\item Run a regression where the outcome variable is \texttt{voteshare} and the explanatory variable is \texttt{presvote}.

	\lstinputlisting[language=R, firstline=88, lastline=88]{PS3_FOS.R}  
	\begin{verbatim}
	Coefficients:
	(Intercept)     presvote  
			0.4413       0.3880 
	\end{verbatim}
	\vspace{2cm}
	\item Make a scatterplot of the two variables and add the regression line. 	
	\lstinputlisting[language=R, firstline=89, lastline=90]{PS3_FOS.R} 
	\begin{figure}[h!]
		\caption{\footnotesize{Model 3}}
		\vspace{.5cm}
		\centering
		\label{fig:Model 3}			
		\includegraphics[width=1.1\textwidth]{model3.png}
	\end{figure}		
	\vspace{2cm}
	\item Write the prediction equation.
	\lstinputlisting[language=R, firstline=102, lastline=102]{PS3_FOS.R} 
	\end{enumerate}

\newpage	
\section*{Question 4}
\noindent The residuals from part (a) tell us how much of the variation in \texttt{voteshare} is $not$ explained by the difference in spending between incumbent and challenger. The residuals in part (b) tell us how much of the variation in \texttt{presvote} is $not$ explained by the difference in spending between incumbent and challenger in the district.
	\begin{enumerate}
		\item Run a regression where the outcome variable is the residuals from Question 1 and the explanatory variable is the residuals from Question 2.	
		\lstinputlisting[language=R, firstline=105, lastline=105]{PS3_FOS.R} 
		\begin{verbatim}
		Coefficients:
		(Intercept)         res2  
				-1.942e-18    2.569e-01  
		\end{verbatim}
		\vspace{2cm}
		\item Make a scatterplot of the two residuals and add the regression line. 	
		\lstinputlisting[language=R, firstline=106, lastline=107]{PS3_FOS.R} 
		\begin{figure}[h!]
			\caption{\footnotesize{Model 4}}
			\vspace{.5cm}
			\centering
			\label{fig:Model 4}			
			\includegraphics[width=1.1\textwidth]{model4.png}
		\end{figure}		
		\vspace{2cm}
		\item Write the prediction equation.
		\lstinputlisting[language=R, firstline=119, lastline=119]{PS3_FOS.R} 
	\end{enumerate}
	
	\newpage	

\section*{Question 5}
\noindent What if the incumbent's vote share is affected by both the president's popularity and the difference in spending between incumbent and challenger? 
	\begin{enumerate}
		\item Run a regression where the outcome variable is the incumbent's \texttt{voteshare} and the explanatory variables are \texttt{difflog} and \texttt{presvote}.	
		\lstinputlisting[language=R, firstline=122, lastline=126]{PS3_FOS.R} 
		\begin{verbatim}
		Coefficients:
		(Intercept)      difflog     presvote  
				0.44864      0.03554      0.25688  
		\end{verbatim}
		\begin{figure}[h!]
			\caption{\footnotesize{Model 5}}
			\vspace{.5cm}
			\centering
			\label{fig:Model 5}			
			\includegraphics[width=1.1\textwidth]{model5.png}
		\end{figure}		
		\vspace{5cm}
		\item Write the prediction equation.	
		\lstinputlisting[language=R, firstline=132, lastline=132]{PS3_FOS.R} 
		\vspace{5cm}
		\item What is it in this output that is identical to the output in Question 4? Why do you think this is the case?
		\vspace{1cm}
						
	\end{enumerate}
\vspace{3cm}
		Upon closer inspection of the coefficients found for Question 5 and Question 4, we can see that the coefficient modifying \texttt{presvote} in Question 5, 0.25688, is identifical to the coefficient modifying the residuals of Question 2, i.e. the variance in \texttt{presvote} not explained by \texttt{difflog}. This is because this coefficient captures the relationship between variance in \texttt{voteshare} and \texttt{presvote} that cannot be explained by/is independent of \texttt{difflog}. This makes sense in the context of our multiple linear regression equation, where this coefficient should have nothing to do with \texttt{difflog}.
		After digging, I discovered this is true because of the Frisch-Waugh-Lovell theorem, often used in Econometrics. 


\end{document}
